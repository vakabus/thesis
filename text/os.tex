\chapter{Cluster installation instructions}
\label{chap:install}

Start by preparing 3 standard Fedora 38 installations, all in a single LAN, ideally with 2 network cards as described in \cref{sec:hw-env}. All systems should have an accessible root shell. Any time we write about running a command, we always assume it is run from the root shell.

In the attachment, there are two scripts prefixed with \ident{setup}. First, we advise editing both of the scripts:

\begin{itemize}
    \item Edit the \ident{setup1-general.sh} script and change the network configuration function, \ident{configure\_systemd\_networkd}, to match your network environment. Especially the names of your network interfaces.

    \item Edit the \ident{setup2-master.sh} script and change \ident{\$IMAGE} variable with the qualified name of your OVS container (instructions on how to build it in \cref{chap:ovs-mod}).
\end{itemize}

Once you changed the script, follow these steps:

\begin{enumerate}
    \item Run the \ident{setup1-general.sh} script on all machines. Provide \kb{1}, \kb{2} or \kb{3} as an argument to set the hostnames.

    \item Wait for all the systems to reboot.

    \item On the machine you chose as \kb{1}, run the \ident{setup2-master.sh} script.

    \item On \kb{1}, run \ident{kubeadm token create -{}-print-join-command}

    \item Append the \ident{-{}-cri-socket=unix:///var/run/cri-dockerd.sock} option at the end of the output of the previous step and run the command on both \kb{2} and \kb{3}.

    \item Wait for the cluster to initialize, i.e. the command \ident{kubectl get nodes} on \kb{1} should shows that all 3 nodes are in the \ident{Ready} state.

    \item Copy pod specification files from the \ident{kube\_configs} directory to \kb{1}. Edit the \ident{reflector.yaml} file and change the image reference to your build of the \ident{analyzer} container (see \cref{chap:analyzer}). Create the pods by calling \ident{kubectl create -f \$file}.

    \item Check the pods deployment status using \ident{kubectl get pods}. Make sure that all pods are in the \ident{Ready} state before proceeding.

    \item At this point, you can run the experiments. To get a root shell in the \ident{arch} pod, use the \ident{kubectl exec -ti arch -{}- /bin/bash} command. Similarly for the \ident{victim} pod.
\end{enumerate}