\chapter{\ident{ovs-vswitchd} container image}
\label{chap:ovs-mod}

\section{Build}

There is a script \ident{build\_ovs\_container.sh} attached to this thesis. Running this script will download and compile OVS, OVN and OVN-Kubernetes and at the end, wrap the result into a container usable for deployment into a cluster. We recommend reading the script first before running it. It is not long and it might be desirable to tweak it.

To run the script, make sure that you have these dependencies installed on your system:

\begin{itemize}
    \item \ident{podman}, ideally configured for rootless operation
    \item \ident {git}
    \item \ident{go} compiler
\end{itemize}

The script creates a container image by default called \ident{ovn-kube-f:latest}. To use it further in the cluster, push it to an accessible container registry\footnote{We run a private registry for this purpose.} and note its fully qualified name.

\section{Our changes to OVS}

All changes we have made to OVS can be seen in the \ident{ovs-usdt-probes.patch} file. We have added only new USDT probes. They are compiled only as \ident{nop} instructions and should not significantly impact the performance unless used.
