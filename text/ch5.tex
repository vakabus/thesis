\chapter{Notable accidental discoveries and dead-ends}
\label{chap:accidents}

\xxx{This chapter might not be included in the final version. Now it's just a placeholder with a dump of random information. If you are reviewing my work, please jump straight away to the next chapter.}

\section{Accidental bug discovery}
\todo{this is really a placeholder with the email I wrote, replace it with something meaningful...}

Hi,

I think I might have found a bug or just a really weird unexpected behavior. I am reporting it here, but I've reproduced it only on systems with OVN-Kubernetes and OVN. I don't know whether it can be replicated with standalone OVS.


\subsection{Test environment}


3 node Kubernetes cluster using OVN-Kubernetes, Docker and Fedora 38. All nodes in a single LAN with IPs 192.168.1.{221,222,223}. From now on, I will refer to the nodes only by the last digit of their IP address.


\subsection{Steps to reproduce:}


\begin{enumerate}
\item install a pod with a shell and a ping command (I am using docker.io/archlinux:latest, always running on node 222)
\item run `kubectl exec -ti \$POD\_NAME -- ping 192.168.1.221`
\item in a different terminal, ssh into the host of the pod (in my case node 222) and run `ovs-dpctl del-flows`
\item observe the measured latencies
\item keep the ping running and kill ovs-vswitchd on the host and wait for its restart
\item observe the latencies
\end{enumerate}


\subsection{What I observe}


\begin{verbatim}
[root@wsfd-netdev64 ~]# kubectl exec -ti arch -- ping 192.168.1.221
PING 192.168.1.221 (192.168.1.221) 56(84) bytes of data.
64 bytes from 192.168.1.221: icmp_seq=1 ttl=63 time=0.543 ms
64 bytes from 192.168.1.221: icmp_seq=2 ttl=63 time=0.160 ms
64 bytes from 192.168.1.221: icmp_seq=3 ttl=63 time=0.119 ms
64 bytes from 192.168.1.221: icmp_seq=4 ttl=63 time=0.144 ms
64 bytes from 192.168.1.221: icmp_seq=5 ttl=63 time=0.137 ms
64 bytes from 192.168.1.221: icmp_seq=6 ttl=63 time=0.996 ms  # < ovs-dpctl del-flows
64 bytes from 192.168.1.221: icmp_seq=7 ttl=63 time=0.808 ms
64 bytes from 192.168.1.221: icmp_seq=8 ttl=63 time=1.01 ms
64 bytes from 192.168.1.221: icmp_seq=9 ttl=63 time=1.24 ms
64 bytes from 192.168.1.221: icmp_seq=10 ttl=63 time=1.20 ms
64 bytes from 192.168.1.221: icmp_seq=11 ttl=63 time=1.14 ms
64 bytes from 192.168.1.221: icmp_seq=12 ttl=63 time=1.10 ms  # < killall ovs-vswitchd
  From 10.244.1.5 icmp_seq=22 Destination Host Unreachable
  From 10.244.1.5 icmp_seq=23 Destination Host Unreachable
  From 10.244.1.5 icmp_seq=24 Destination Host Unreachable
  From 10.244.1.5 icmp_seq=25 Destination Host Unreachable
  From 10.244.1.5 icmp_seq=26 Destination Host Unreachable
  From 10.244.1.5 icmp_seq=27 Destination Host Unreachable
  From 10.244.1.5 icmp_seq=28 Destination Host Unreachable
  From 10.244.1.5 icmp_seq=29 Destination Host Unreachable
  From 10.244.1.5 icmp_seq=31 Destination Host Unreachable
  From 10.244.1.5 icmp_seq=32 Destination Host Unreachable
64 bytes from 192.168.1.221: icmp_seq=34 ttl=63 time=1371 ms
64 bytes from 192.168.1.221: icmp_seq=35 ttl=63 time=322 ms
64 bytes from 192.168.1.221: icmp_seq=36 ttl=63 time=0.186 ms
64 bytes from 192.168.1.221: icmp_seq=37 ttl=63 time=0.192 ms
64 bytes from 192.168.1.221: icmp_seq=38 ttl=63 time=0.140 ms
64 bytes from 192.168.1.221: icmp_seq=39 ttl=63 time=0.163 ms
^C
--- 192.168.1.221 ping statistics ---
39 packets transmitted, 18 received, +10 errors, 53.8462% packet loss, time 38769ms
rtt min/avg/max/mdev = 0.119/94.570/1370.551/318.102 ms, pipe 3
\end{verbatim}

After flushing the datapath flow rules, the RTTs increase. This indicates an upcall for every single packet. I confirmed this by tracing the kernel and the upcalls are definitely there (actually, this is how I discovered the whole issue). The system can remain in this state for a long time (at least minutes).

After restarting vswitchd, the upcalls stop happening and everything is fast again.


I have also confirmed, that the same thing happens to UDP packets. The same also happens regardless of which node in the cluster I target, even the pod's host.


\subsection{What I expect}


I expected only a few upcalls in response to flushing the datapath, after which a new datapath rules would get inserted into the kernel.


\subsection{Repeatability}


Flushing the rules and restarting vswitchd seems to be a certain method to flip the system between the two different states. However the upcall-making state seems unstable and it sometimes reverts back to the expected lower-latency state by itself. It appears to me that flooding the system with unrelated upcalls and rules increases the chance of an "autonomous fix".


At this point, I don't believe this would cause significant problems. I just find the behavior extremely weird.

Is there something more I could provide to help with reproducing this? Or should I report it elsewhere?


Best,
Vašek Šraier 

\section{Lock contention investigation}
\todo{write about the fact, that there does not seem to be a problematic lock in OVS, because we checked them almost all}


