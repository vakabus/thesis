\chapter{\ident{analyzer} - the tool for running experiments}
\label{chap:analyzer}

\section{Build}

The source code is located in the \ident{analyzer} directory in the attachment. Only the Rust toolchain is required for compilation:

\begin{verbatim}
# in the analyzer/ directory
cargo build --release --target=x86_64-unknown-linux-musl
# the executable will be located at 
#    analyzer/target/x86_64-unknown-linux-musl/release/analyzer
\end{verbatim}

Continue by building the container image:

\begin{verbatim}
# again in the analyzer/ directory
podman build -t analyzer .
\end{verbatim}

Push the image to your container registry of choice and remember the qualified container name. The image is required for the \ident{reflector} pod, more in \cref{chap:install}. Also, copy the binary to the \ident{\$PATH} on all nodes and pods.

\section{Usage}

The \ident{analyzer} is generally a collection of smaller tools, all invoked via subcommands. The tool can be always supplied with the \ident{-{}-help} option and it will print out a help message. The following subsections will describe how to use the \ident{analyzer} for the discussed experiments.

\paragraph{Data collection} The measurement results are stored in files in the \ident{analyzer}'s current working directory. Multiple files are usually created. The file names start with a human-readable identifier of the data series. The second part of the file names is a timestamp (identical for all files created in a single \ident{analyzer} run).

The \ident{analyzer} can upload results to an HTTP server when provided with the \ident{-{}-push-results-url} argument. When the URL is provided, for every data file it creates, the \ident{analyzer} calls the \ident{curl -T [FILE] \{URL\}} command. 

\paragraph{Data format}

The result files are usually using the CSV format. All data series use the same time source (see \cref{subsec:clock}). Therefore, all measurements can be easily aligned even when the collected data came from different \ident{analyzer} runs (i.e. in a pod and on a host at the same time).

\subsection{Eviction timeout measurement}

The result of this experiment is a single CSV file with timestamps and the measured round-trip times.

\begin{verbatim}
# on arch (pod)
analyzer randomized-eviction-timeout \
		--target-ip 192.168.1.221 \
		--count 3
\end{verbatim}

\subsection{Packet fuzzing}

This experiment creates multiple data files, not all of which are relevant. From the \ident{node-logger} subcommand, we are interested in upcall statistics located in the \ident{kernel\_flow\_table\_trace*.csv} file. The \ident{tags*.jsonl} file from the \ident{packet-fuzz} subcommand provides us with timestamps allowing us to separate the upcalls into categories.

\begin{verbatim}
# on kb2
analyzer install-dependencies
analyzer node-logger --only-upcalls

# on arch (pod)
analyzer install-dependencies
analyzer packet-fuzz
\end{verbatim}


\subsection{Packet flood}

In this experiment, we were not looking for anything in particular, therefore all the result files can be of interest.

\begin{verbatim}
# on arch (pod)
analyzer packet-flood --count 500000

# on victim (pod)
analyzer victim

# on kb2
analyzer install-dependencies
analyzer node-logger --only-upcalls
\end{verbatim}

\section{Processing of results}

Scripts for creating plots are located in the \ident{analyzer/postprocessing} directory. To use them, it is necessary to install Python, matplotlib, Polars, Pandas, numpy and scipy.

There is a script for every experiment type. The packet flood experiment has multiple similar scripts with slightly differently preconfigured plots, the others have only a single script. The scripts, especially the packet-flood related, contain extra commented-out code to plot additional data.

All scripts expect a single argument - a path to a directory containing all results from a single experiment.